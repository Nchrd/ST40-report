\chapter{Environnement de travail}
\section{UTBM}
\paragraph{} \hspace{10mm}
L'UTBM (Université de Technologie de Belfort-Montbéliard) est une université française située dans le département du Territoire de Belfort, dans la région Bourgogne-Franche-Comté. Elle est spécialisée dans les domaines de l'ingénierie et de la technologie, et propose des formations de bachelor, de master et de doctorat dans ces domaines. L'UTBM compte environ 2 800 étudiants et dispose de plusieurs campus à Belfort, Sevenans et Montbéliard.

\begin{figure} [H]
    \centering
    \includegraphics[width=0.4\textwidth]{assets/logo/utbm_logo.png}
    \caption{Logo de l'UTBM}
    \label{fig:logoUTBM}
\end{figure}

\section{Femto-ST}
\paragraph{} \hspace{10mm}
Le laboratoire FEMTO-ST (Franche-Comté Électronique Mécanique Thermique et Optique - Sciences et Technologies) est un laboratoire de recherche en sciences et technologies situé à Besançon, en France. Il est affilié à l'université de Technologie de Belfort-Montbéliard (UTBM) et à l'université de Franche-Comté (UFC). Le laboratoire FEMTO-ST a pour objectif de mener des recherches dans les domaines de l'électronique, de la mécanique, de la thermique et de l'optique, et de développer des technologies innovantes dans ces domaines. Il dispose de plusieurs équipes de recherche spécialisées dans différents domaines, comme la micro-nanotechnologie, les systèmes embarqués, l'informatique et les réseaux, ou encore la mécanique et la thermique. Le laboratoire FEMTO-ST travaille en collaboration avec de nombreux partenaires industriels et universitaires, et est reconnu pour la qualité de ses recherches et de ses contributions scientifiques dans ces domaines.
\begin{figure} [H]
    \centering
    \includegraphics[width=0.4\textwidth]{assets/logo/femto-st.png}
    \caption{Logo du laboratoire FEMTO-ST}
    \label{fig:logoFemto}
\end{figure}

\section{RECITS}
\paragraph{} \hspace{10mm}
Le projet RECITS (Recherche et étude sur le changement industriel, technologique et sociétal) est un projet fondé en 2000 à \textbf{l'Université de technologie de Belfort-Montbéliard} et faisant partie du laboratoire \textbf{FEMTO-ST}. Il se distingue par trois spécificités dans le paysage de la recherche en sciences humaines et sociales en France : un objet de recherche original visant à comprendre la relation entre l'homme et la technologie, une composition pluridisciplinaire pour une meilleure compréhension de cet objet complexe, et une insertion dans un environnement de recherche en sciences pour l'ingénieur. La philosophie générale de l'équipe est de favoriser un approfondissement disciplinaire préalable à l'élargissement interdisciplinaire, avec une expertise commune originale sur les techniques, la conception et le changement technique. 


\section{Service d'accueil}
\paragraph{} \hspace{10mm}
J'ai réalisé ces 6 mois de stage dans le bâtiment H de l'UTBM, où est installée la direction du Département Informatique. Cela m'a permis d'avoir accès à tout le matériel nécessaire tel que des logiciels, des ordinateurs à la puissance adaptée, ainsi que deux serveurs de développement pour y héberger une base de données relationnelle et une base de données ontologique.

Cette proximité avec l'UTBM m'a également donné la possibilité de facilement rester en contact avec mon enseignant suiveur M.Gechter.