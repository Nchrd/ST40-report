\section{Sujet de stage}
\subsection{Sujet initial}
\paragraph{} \hspace{10mm}
Le sujet de stage proposé consistait à enrichir et mettre en place des outils de manipulation de \textbf{la base de données TTM}. Cela incluait l'ajout de nouvelles données et la création de requêtes spécifiques.

Ensuite, nous devions définir les ontologies sous \textbf{OntoMe} et les stocker sous \textbf{Omeka-S}. Cela impliquait la création de modèles de données structurés pour représenter les concepts et les relations entre eux.

Une partie importante de ce stage était la mise en place d'une interface permettant de faire le lien entre la TTM BDD existante et les ontologies. Cela nécessitait une formalisation précise des données et des relations entre elles.

Enfin, nous devions mis en place un protocole d'échanges entre les outils TTM et Omeka-S afin de rendre ces outils inter-opérables. Cela aurait permis de faciliter l'intégration des données et de rendre les outils plus efficaces dans leur utilisation.
\subsection{Sujet réel}
\paragraph{} \hspace{10mm}
Le sujet réel a porté dans un premier temps sur l'amélioration de la base de données existante et du site web. Cela a impliqué l'optimisation des formulaires et l'ajout de fonctionnalités pour faciliter l'utilisation du site ainsi que l'amélioration du modèle relationnel du projet.

En outre, j'ai également travaillé sur la définition d'une ontologie personnalisée basée sur celle du CIDOC-CRM. Cela a été réalisé suite à mon séjour à Nantes, où j'ai pu en apprendre davantage sur les différents concepts et terminologies utilisés dans l'ontologie.

J'ai ensuite validé ce modèle ontologique à l'aide de Protégé et de la spécification SHACL et effectué la première saisie de données.

Enfin, j'ai réalisé une représentation en 3D sous Unity d'un flux à l'aide des données saisies via l'API d'Omeka-S.